%%%%%%%%%%%%%%%%%%%%%%%%%%%%%%%%%%%%%%%%%%%%%%%%%%%%%%%%%%%%%%%%%%%%%%%%%%%%%%%
% Chapter 'Adsorption - R-507a - activated carbon powder Maxsorb III'
%%%%%%%%%%%%%%%%%%%%%%%%%%%%%%%%%%%%%%%%%%%%%%%%%%%%%%%%%%%%%%%%%%%%%%%%%%%%%%%
\subsection{Activated carbon powder Maxsorb III}
%
%%%%%%%%%%%%%%%%%%%%%%%%%%%%%%%%%%%%%%%%%%%%%%%%%%%%%%%%%%%%%%%%%%%%%%%%%%%%%%%
%%%%%%%%%%%%%%%%%%%%%%%%%%%%%%%%%%%%%%%%%%%%%%%%%%%%%%%%%%%%%%%%%%%%%%%%%%%%%%%
\subsubsection{DubininAstakhov - ID 1}
%
\begin{tabular}[l]{|lp{11.5cm}|}
\hline
\addlinespace

\textbf{Sorbent:} & activated carbon powder \\
\textbf{Subtype:} & Maxsorb III \\
\textbf{Refrigerant:} & R-507a \\
\textbf{Equation:} & DubininAstakhov \\
\textbf{ID:} & 1 \\
\textbf{Reference:} & Saha, Bidyut B.; El-Sharkawy, Ibrahim I.; Habib, Khairul; Koyama, Shigeru; Srinivasan, Kandadai (2008): Adsorption of Equal Mass Fraction Near an Azeotropic Mixture of Pentafluoroethane and 1,1,1-Trifluoroethane on Activated Carbon. In: J. Chem. Eng. Data 53 (8), S. 1872–1876. DOI: 10.1021/je800204p. \\
\textbf{Comment:} & None \\

\addlinespace
\hline
\end{tabular}
\newline

\textbf{Properties of sorbent:}
\newline
%
\begin{longtable}[l]{lll}
\toprule
\addlinespace
\textbf{Property} & \textbf{Unit} & \textbf{Value} \\
\addlinespace
\midrule
\endhead
\bottomrule
\endfoot
\bottomrule
\endlastfoot
\addlinespace

Diameter of pellet & \si{\milli\meter} & 0.072\\
Surface area & \si{\square\meter\per\gram} & 3142\\
Pore volume & \si{\milli\cubic\meter\per\gram} & 1.7\\

\addlinespace\end{longtable}

\textbf{Equation and parameters:}
\newline
%
Loading $w$ in $\si{\kilogram\per\kilogram}$ is calculated depending on pressure $p$ in $\si{\pascal}$, temperature $T$ in $\si{\kelvin}$, and vapor pressure $p_\mathrm{sat}$ in $\si{\pascal}$ by:
%
\begin{equation*}
\begin{split}
w &=& \begin{cases} W \rho_\mathrm{sat}^{\mathrm{liq}} & \quad \text{if flag} \geq 0 \\ W & \quad \text{else} \end{cases} & \quad\text{, and} \\
W &=& W_0 \exp \left( - \left( \nicefrac{A}{E} \right) ^{n} \right) & \quad\text{, and} \\
A &=& R T \ln \left( \nicefrac{p_\mathrm{sat}}{p} \right) & \quad\text{.} \\
\end{split}
\end{equation*}
%
The parameters of the equation are:
%
\begin{longtable}[l]{lll|lll}
\toprule
\addlinespace
\textbf{Par.} & \textbf{Unit} & \textbf{Value} &	\textbf{Par.} & \textbf{Unit} & \textbf{Value} \\
\addlinespace
\midrule
\endhead

\bottomrule
\endfoot
\bottomrule
\endlastfoot
\addlinespace

flag & - & 1.000000000e+00 & $n$ & - & 1.470000000e+00 \\
$E$ & $\si{\joule\per\mole}$ & 5.740000000e+03 & $W_0$ & $\si{\cubic\meter\per\kilogram}$ & 1.175000000e-03 \\

\addlinespace\end{longtable}

\textbf{Validity:}
\newline
No data on validity available!
\newline

\textbf{Visualization:}
%
\newline
No experimental data exists. Thus, isotherm is not visualized!
%

\FloatBarrier
\newpage
%%%%%%%%%%%%%%%%%%%%%%%%%%%%%%%%%%%%%%%%%%%%%%%%%%%%%%%%%%%%%%%%%%%%%%%%%%%%%%%
%%%%%%%%%%%%%%%%%%%%%%%%%%%%%%%%%%%%%%%%%%%%%%%%%%%%%%%%%%%%%%%%%%%%%%%%%%%%%%%
\subsubsection{DubininAstakhov - ID 2}
%
\begin{tabular}[l]{|lp{11.5cm}|}
\hline
\addlinespace

\textbf{Sorbent:} & activated carbon powder \\
\textbf{Subtype:} & Maxsorb III \\
\textbf{Refrigerant:} & R-507a \\
\textbf{Equation:} & DubininAstakhov \\
\textbf{ID:} & 2 \\
\textbf{Reference:} & Loh, Wai Soong; Ismail, Azhar Bin; Xi, Baojuan; Ng, Kim Choon; Chun, Won Gee (2012): Adsorption Isotherms and Isosteric Enthalpy of Adsorption for Assorted Refrigerants on Activated Carbons. In: J. Chem. Eng. Data 57 (10), S. 2766–2773. DOI: 10.1021/je3008099. \\
\textbf{Comment:} & None \\

\addlinespace
\hline
\end{tabular}
\newline

\textbf{Properties of sorbent:}
\newline
%
\begin{longtable}[l]{lll}
\toprule
\addlinespace
\textbf{Property} & \textbf{Unit} & \textbf{Value} \\
\addlinespace
\midrule
\endhead
\bottomrule
\endfoot
\bottomrule
\endlastfoot
\addlinespace

Diameter of pore & \si{\milli\meter} & 0.000002\\
Surface area & \si{\square\meter\per\gram} & 3150\\
Pore volume & \si{\milli\cubic\meter\per\gram} & 1.7\\
Solid density & \si{\kilogram\per\cubic\meter} & 2200\\

\addlinespace\end{longtable}

\textbf{Equation and parameters:}
\newline
%
Loading $w$ in $\si{\kilogram\per\kilogram}$ is calculated depending on pressure $p$ in $\si{\pascal}$, temperature $T$ in $\si{\kelvin}$, and vapor pressure $p_\mathrm{sat}$ in $\si{\pascal}$ by:
%
\begin{equation*}
\begin{split}
w &=& \begin{cases} W \rho_\mathrm{sat}^{\mathrm{liq}} & \quad \text{if flag} \geq 0 \\ W & \quad \text{else} \end{cases} & \quad\text{, and} \\
W &=& W_0 \exp \left( - \left( \nicefrac{A}{E} \right) ^{n} \right) & \quad\text{, and} \\
A &=& R T \ln \left( \nicefrac{p_\mathrm{sat}}{p} \right) & \quad\text{.} \\
\end{split}
\end{equation*}
%
The parameters of the equation are:
%
\begin{longtable}[l]{lll|lll}
\toprule
\addlinespace
\textbf{Par.} & \textbf{Unit} & \textbf{Value} &	\textbf{Par.} & \textbf{Unit} & \textbf{Value} \\
\addlinespace
\midrule
\endhead

\bottomrule
\endfoot
\bottomrule
\endlastfoot
\addlinespace

flag & - & -1.000000000e+00 & $n$ & - & 1.340000000e+00 \\
$E$ & $\si{\joule\per\mole}$ & 7.547250000e+03 & $W_0$ & $\si{\kilogram\per\kilogram}$ & 2.050000000e+00 \\

\addlinespace\end{longtable}

\textbf{Validity:}
\newline
Equation is approximately valid for $4827.0 \si{\pascal} \leq p \leq 1067573.0 \si{\pascal}$,  $278.15 \si{\kelvin} \leq T \leq 318.15 \si{\kelvin}$, and $0.19899 \si{\kilogram\per\kilogram} \leq w \leq 2.06674 \si{\kilogram\per\kilogram}$.
\newline

\textbf{Visualization:}
%
\begin{figure}[!htp]
{\noindent\includegraphics[height=10cm, keepaspectratio]{figs/ads/ads_R-507a_activated_carbon_powder_Maxsorb_III_DubininAstakhov_2.pdf}}
\end{figure}
%

To generate the figure, the following refrigerant functions were selected:
\begin{itemize}
\item Vapor pressure: VaporPressure\_EoS1 - ID 1
\item Saturated liquid density: NoSaturatedLiquidDensity - ID 1
\end{itemize}

The uncertainity of the experimental data is:
\begin{itemize}
\item Data source $\,\to\,$ Data was taken from figure
\item Pressure, relative, in \% $\,\to\,$ 0 Pa < p < 5 MPa: 0.1% full scale
\item Temperature, absolute, in $\si{\kelvin}$ $\,\to\,$ 0.15
\end{itemize}

The mean absolute percentage error (MAPE) between the experimental and calculated data results in 212.94\%.
\FloatBarrier
\newpage
%%%%%%%%%%%%%%%%%%%%%%%%%%%%%%%%%%%%%%%%%%%%%%%%%%%%%%%%%%%%%%%%%%%%%%%%%%%%%%%
