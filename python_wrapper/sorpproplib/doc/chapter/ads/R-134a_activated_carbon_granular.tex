%%%%%%%%%%%%%%%%%%%%%%%%%%%%%%%%%%%%%%%%%%%%%%%%%%%%%%%%%%%%%%%%%%%%%%%%%%%%%%%
% Chapter 'Adsorption - R-134a - activated carbon granular'
%%%%%%%%%%%%%%%%%%%%%%%%%%%%%%%%%%%%%%%%%%%%%%%%%%%%%%%%%%%%%%%%%%%%%%%%%%%%%%%
\subsection{Activated carbon granular}
%
%%%%%%%%%%%%%%%%%%%%%%%%%%%%%%%%%%%%%%%%%%%%%%%%%%%%%%%%%%%%%%%%%%%%%%%%%%%%%%%
%%%%%%%%%%%%%%%%%%%%%%%%%%%%%%%%%%%%%%%%%%%%%%%%%%%%%%%%%%%%%%%%%%%%%%%%%%%%%%%
\subsubsection{DubininAstakhov - ID 1}
%
\begin{tabular}[l]{|lp{11.5cm}|}
\hline
\addlinespace

\textbf{Sorbent:} & activated carbon \\
\textbf{Subtype:} & granular \\
\textbf{Refrigerant:} & R-134a \\
\textbf{Equation:} & DubininAstakhov \\
\textbf{ID:} & 1 \\
\textbf{Reference:} & Askalany, Ahmed A.; Salem, M.; Ismail, I. M.; Ali, Ahmed Hamza H.; Morsy, M. G. (2012): Experimental study on adsorption–desorption characteristics of granular activated carbon/R134a pair. In: International Journal of Refrigeration 35 (3), S. 494–498. DOI: 10.1016/j.ijrefrig.2011.04.002. \\
\textbf{Comment:} & None \\

\addlinespace
\hline
\end{tabular}
\newline

\textbf{Properties of sorbent:}
\newline
%
\begin{longtable}[l]{lll}
\toprule
\addlinespace
\textbf{Property} & \textbf{Unit} & \textbf{Value} \\
\addlinespace
\midrule
\endhead
\bottomrule
\endfoot
\bottomrule
\endlastfoot
\addlinespace

Diameter of pore & \si{\milli\meter} & 0.000002\\
Surface area & \si{\square\meter\per\gram} & 900\\
Pore volume & \si{\milli\cubic\meter\per\gram} & 0.88\\
Bulk density & \si{\kilogram\per\cubic\meter} & 510\\

\addlinespace\end{longtable}

\textbf{Equation and parameters:}
\newline
%
Loading $w$ in $\si{\kilogram\per\kilogram}$ is calculated depending on pressure $p$ in $\si{\pascal}$, temperature $T$ in $\si{\kelvin}$, and vapor pressure $p_\mathrm{sat}$ in $\si{\pascal}$ by:
%
\begin{equation*}
\begin{split}
w &=& \begin{cases} W \rho_\mathrm{sat}^{\mathrm{liq}} & \quad \text{if flag} \geq 0 \\ W & \quad \text{else} \end{cases} & \quad\text{, and} \\
W &=& W_0 \exp \left( - \left( \nicefrac{A}{E} \right) ^{n} \right) & \quad\text{, and} \\
A &=& R T \ln \left( \nicefrac{p_\mathrm{sat}}{p} \right) & \quad\text{.} \\
\end{split}
\end{equation*}
%
The parameters of the equation are:
%
\begin{longtable}[l]{lll|lll}
\toprule
\addlinespace
\textbf{Par.} & \textbf{Unit} & \textbf{Value} &	\textbf{Par.} & \textbf{Unit} & \textbf{Value} \\
\addlinespace
\midrule
\endhead

\bottomrule
\endfoot
\bottomrule
\endlastfoot
\addlinespace

flag & - & -1.000000000e+00 & $n$ & - & 1.830000000e+00 \\
$E$ & $\si{\joule\per\mole}$ & 9.575000000e+03 & $W_0$ & $\si{\kilogram\per\kilogram}$ & 1.680000000e+00 \\

\addlinespace\end{longtable}

\textbf{Validity:}
\newline
Equation is approximately valid for $293.15 \si{\kelvin} \leq T \leq 363.15 \si{\kelvin}$.
\newline

\textbf{Visualization:}
%
\newline
No experimental data exists. Thus, isotherm is not visualized!
%

\FloatBarrier
\newpage
%%%%%%%%%%%%%%%%%%%%%%%%%%%%%%%%%%%%%%%%%%%%%%%%%%%%%%%%%%%%%%%%%%%%%%%%%%%%%%%
