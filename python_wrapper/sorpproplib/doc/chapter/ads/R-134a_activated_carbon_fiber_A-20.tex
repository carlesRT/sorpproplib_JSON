%%%%%%%%%%%%%%%%%%%%%%%%%%%%%%%%%%%%%%%%%%%%%%%%%%%%%%%%%%%%%%%%%%%%%%%%%%%%%%%
% Chapter 'Adsorption - R-134a - activated carbon fiber A-20'
%%%%%%%%%%%%%%%%%%%%%%%%%%%%%%%%%%%%%%%%%%%%%%%%%%%%%%%%%%%%%%%%%%%%%%%%%%%%%%%
\subsection{Activated carbon fiber A-20}
%
%%%%%%%%%%%%%%%%%%%%%%%%%%%%%%%%%%%%%%%%%%%%%%%%%%%%%%%%%%%%%%%%%%%%%%%%%%%%%%%
%%%%%%%%%%%%%%%%%%%%%%%%%%%%%%%%%%%%%%%%%%%%%%%%%%%%%%%%%%%%%%%%%%%%%%%%%%%%%%%
\subsubsection{DubininAstakhov - ID 1}
%
\begin{tabular}[l]{|lp{11.5cm}|}
\hline
\addlinespace

\textbf{Sorbent:} & activated carbon fiber \\
\textbf{Subtype:} & A-20 \\
\textbf{Refrigerant:} & R-134a \\
\textbf{Equation:} & DubininAstakhov \\
\textbf{ID:} & 1 \\
\textbf{Reference:} & Saha, Bidyut Baran; El-Sharkawy, Ibrahim I.; Thorpe, Roger; Critoph, Robert E. (2012): Accurate adsorption isotherms of R134a onto activated carbons for cooling and freezing applications. In: International Journal of Refrigeration 35 (3), S. 499–505. DOI: 10.1016/j.ijrefrig.2011.05.002. \\
\textbf{Comment:} & See original literature: Use low-level interface for calculations as special form of density of adsorpt is required (i.e., 1/rho\_adsorpt = 7.2643e-4 * exp(ln(9.39e-4 /7.2643e-4) * (T - 246.78) / (374.21 - 246.78)) in kg/m3); inverse functions may not work anymore. \\

\addlinespace
\hline
\end{tabular}
\newline

\textbf{Properties of sorbent:}
\newline
%
Property data of sorbent and subtype does not exist.

\textbf{Equation and parameters:}
\newline
%
Loading $w$ in $\si{\kilogram\per\kilogram}$ is calculated depending on pressure $p$ in $\si{\pascal}$, temperature $T$ in $\si{\kelvin}$, and vapor pressure $p_\mathrm{sat}$ in $\si{\pascal}$ by:
%
\begin{equation*}
\begin{split}
w &=& \begin{cases} W \rho_\mathrm{sat}^{\mathrm{liq}} & \quad \text{if flag} \geq 0 \\ W & \quad \text{else} \end{cases} & \quad\text{, and} \\
W &=& W_0 \exp \left( - \left( \nicefrac{A}{E} \right) ^{n} \right) & \quad\text{, and} \\
A &=& R T \ln \left( \nicefrac{p_\mathrm{sat}}{p} \right) & \quad\text{.} \\
\end{split}
\end{equation*}
%
The parameters of the equation are:
%
\begin{longtable}[l]{lll|lll}
\toprule
\addlinespace
\textbf{Par.} & \textbf{Unit} & \textbf{Value} &	\textbf{Par.} & \textbf{Unit} & \textbf{Value} \\
\addlinespace
\midrule
\endhead

\bottomrule
\endfoot
\bottomrule
\endlastfoot
\addlinespace

flag & - & 1.000000000e+00 & $n$ & - & 1.500000000e+00 \\
$E$ & $\si{\joule\per\mole}$ & 8.611656172e+03 & $W_0$ & $\si{\cubic\meter\per\kilogram}$ & 1.010000000e-03 \\

\addlinespace\end{longtable}

\textbf{Validity:}
\newline
Equation is approximately valid for $139320.0 \si{\pascal} \leq p \leq 994090.0 \si{\pascal}$,  $299.95 \si{\kelvin} \leq T \leq 352.35 \si{\kelvin}$, and $0.44932 \si{\kilogram\per\kilogram} \leq w \leq 0.93974 \si{\kilogram\per\kilogram}$.
\newline

\textbf{Visualization:}
%
\begin{figure}[!htp]
{\noindent\includegraphics[height=10cm, keepaspectratio]{figs/ads/ads_R-134a_activated_carbon_fiber_A-20_DubininAstakhov_1.pdf}}
\end{figure}
%

To generate the figure, the following refrigerant functions were selected:
\begin{itemize}
\item Vapor pressure: VaporPressure\_EoS1 - ID 1
\item Saturated liquid density: SaturatedLiquidDensity\_EoS1 - ID 1
\item Special refrigerant functions as described by comment and CoolProp
\end{itemize}

The uncertainity of the experimental data is:
\begin{itemize}
\item Data source $\,\to\,$ Data was taken from figure
\end{itemize}

The mean absolute percentage error (MAPE) between the experimental and calculated data results in 1.43\%.
\FloatBarrier
\newpage
%%%%%%%%%%%%%%%%%%%%%%%%%%%%%%%%%%%%%%%%%%%%%%%%%%%%%%%%%%%%%%%%%%%%%%%%%%%%%%%
%%%%%%%%%%%%%%%%%%%%%%%%%%%%%%%%%%%%%%%%%%%%%%%%%%%%%%%%%%%%%%%%%%%%%%%%%%%%%%%
\subsubsection{DubininAstakhov - ID 2}
%
\begin{tabular}[l]{|lp{11.5cm}|}
\hline
\addlinespace

\textbf{Sorbent:} & activated carbon fiber \\
\textbf{Subtype:} & A-20 \\
\textbf{Refrigerant:} & R-134a \\
\textbf{Equation:} & DubininAstakhov \\
\textbf{ID:} & 2 \\
\textbf{Reference:} & Saha, Bidyut Baran; El-Sharkawy, Ibrahim I.; Thorpe, Roger; Critoph, Robert E. (2012): Accurate adsorption isotherms of R134a onto activated carbons for cooling and freezing applications. In: International Journal of Refrigeration 35 (3), S. 499–505. DOI: 10.1016/j.ijrefrig.2011.05.002. \\
\textbf{Comment:} & See original literature: Use low-level interface for calculations as special form of density of adsorpt is required (i.e., 1/rho\_adsorpt = 7.2643e-4 * exp(ln(9.39e-4 /7.2643e-4) * (T - 246.78) / (374.21 - 246.78)) in kg/m3); inverse functions may not work anymore. \\

\addlinespace
\hline
\end{tabular}
\newline

\textbf{Properties of sorbent:}
\newline
%
Property data of sorbent and subtype does not exist.

\textbf{Equation and parameters:}
\newline
%
Loading $w$ in $\si{\kilogram\per\kilogram}$ is calculated depending on pressure $p$ in $\si{\pascal}$, temperature $T$ in $\si{\kelvin}$, and vapor pressure $p_\mathrm{sat}$ in $\si{\pascal}$ by:
%
\begin{equation*}
\begin{split}
w &=& \begin{cases} W \rho_\mathrm{sat}^{\mathrm{liq}} & \quad \text{if flag} \geq 0 \\ W & \quad \text{else} \end{cases} & \quad\text{, and} \\
W &=& W_0 \exp \left( - \left( \nicefrac{A}{E} \right) ^{n} \right) & \quad\text{, and} \\
A &=& R T \ln \left( \nicefrac{p_\mathrm{sat}}{p} \right) & \quad\text{.} \\
\end{split}
\end{equation*}
%
The parameters of the equation are:
%
\begin{longtable}[l]{lll|lll}
\toprule
\addlinespace
\textbf{Par.} & \textbf{Unit} & \textbf{Value} &	\textbf{Par.} & \textbf{Unit} & \textbf{Value} \\
\addlinespace
\midrule
\endhead

\bottomrule
\endfoot
\bottomrule
\endlastfoot
\addlinespace

flag & - & -1.000000000e+00 & $n$ & - & 1.400000000e+00 \\
$E$ & $\si{\joule\per\mole}$ & 7.720071086e+03 & $W_0$ & $\si{\kilogram\per\kilogram}$ & 1.256000000e+00 \\

\addlinespace\end{longtable}

\textbf{Validity:}
\newline
Equation is approximately valid for $139320.0 \si{\pascal} \leq p \leq 994090.0 \si{\pascal}$,  $299.95 \si{\kelvin} \leq T \leq 352.35 \si{\kelvin}$, and $0.44932 \si{\kilogram\per\kilogram} \leq w \leq 0.93974 \si{\kilogram\per\kilogram}$.
\newline

\textbf{Visualization:}
%
\begin{figure}[!htp]
{\noindent\includegraphics[height=10cm, keepaspectratio]{figs/ads/ads_R-134a_activated_carbon_fiber_A-20_DubininAstakhov_2.pdf}}
\end{figure}
%

To generate the figure, the following refrigerant functions were selected:
\begin{itemize}
\item Vapor pressure: VaporPressure\_EoS1 - ID 1
\item Saturated liquid density: SaturatedLiquidDensity\_EoS1 - ID 1
\item Special refrigerant functions as described by comment and CoolProp
\end{itemize}

The uncertainity of the experimental data is:
\begin{itemize}
\item Data source $\,\to\,$ Data was taken from figure
\end{itemize}

The mean absolute percentage error (MAPE) between the experimental and calculated data results in 3.64\%.
\FloatBarrier
\newpage
%%%%%%%%%%%%%%%%%%%%%%%%%%%%%%%%%%%%%%%%%%%%%%%%%%%%%%%%%%%%%%%%%%%%%%%%%%%%%%%
%%%%%%%%%%%%%%%%%%%%%%%%%%%%%%%%%%%%%%%%%%%%%%%%%%%%%%%%%%%%%%%%%%%%%%%%%%%%%%%
\subsubsection{DubininAstakhov - ID 3}
%
\begin{tabular}[l]{|lp{11.5cm}|}
\hline
\addlinespace

\textbf{Sorbent:} & activated carbon fiber \\
\textbf{Subtype:} & A-20 \\
\textbf{Refrigerant:} & R-134a \\
\textbf{Equation:} & DubininAstakhov \\
\textbf{ID:} & 3 \\
\textbf{Reference:} & Loh, Wai Soong; Ismail, Azhar Bin; Xi, Baojuan; Ng, Kim Choon; Chun, Won Gee (2012): Adsorption Isotherms and Isosteric Enthalpy of Adsorption for Assorted Refrigerants on Activated Carbons. In: J. Chem. Eng. Data 57 (10), S. 2766–2773. DOI: 10.1021/je3008099. \\
\textbf{Comment:} & None \\

\addlinespace
\hline
\end{tabular}
\newline

\textbf{Properties of sorbent:}
\newline
%
\begin{longtable}[l]{lll}
\toprule
\addlinespace
\textbf{Property} & \textbf{Unit} & \textbf{Value} \\
\addlinespace
\midrule
\endhead
\bottomrule
\endfoot
\bottomrule
\endlastfoot
\addlinespace

Diameter of pore & \si{\milli\meter} & 0.00000216\\
Surface area & \si{\square\meter\per\gram} & 1930\\
Pore volume & \si{\milli\cubic\meter\per\gram} & 1.05\\
Solid density & \si{\kilogram\per\cubic\meter} & 2200\\

\addlinespace\end{longtable}

\textbf{Equation and parameters:}
\newline
%
Loading $w$ in $\si{\kilogram\per\kilogram}$ is calculated depending on pressure $p$ in $\si{\pascal}$, temperature $T$ in $\si{\kelvin}$, and vapor pressure $p_\mathrm{sat}$ in $\si{\pascal}$ by:
%
\begin{equation*}
\begin{split}
w &=& \begin{cases} W \rho_\mathrm{sat}^{\mathrm{liq}} & \quad \text{if flag} \geq 0 \\ W & \quad \text{else} \end{cases} & \quad\text{, and} \\
W &=& W_0 \exp \left( - \left( \nicefrac{A}{E} \right) ^{n} \right) & \quad\text{, and} \\
A &=& R T \ln \left( \nicefrac{p_\mathrm{sat}}{p} \right) & \quad\text{.} \\
\end{split}
\end{equation*}
%
The parameters of the equation are:
%
\begin{longtable}[l]{lll|lll}
\toprule
\addlinespace
\textbf{Par.} & \textbf{Unit} & \textbf{Value} &	\textbf{Par.} & \textbf{Unit} & \textbf{Value} \\
\addlinespace
\midrule
\endhead

\bottomrule
\endfoot
\bottomrule
\endlastfoot
\addlinespace

flag & - & -1.000000000e+00 & $n$ & - & 1.490000000e+00 \\
$E$ & $\si{\joule\per\mole}$ & 7.136010000e+03 & $W_0$ & $\si{\kilogram\per\kilogram}$ & 1.290000000e+00 \\

\addlinespace\end{longtable}

\textbf{Validity:}
\newline
Equation is approximately valid for $17485.0 \si{\pascal} \leq p \leq 685047.0 \si{\pascal}$,  $278.15 \si{\kelvin} \leq T \leq 328.15 \si{\kelvin}$, and $0.05756 \si{\kilogram\per\kilogram} \leq w \leq 1.28264 \si{\kilogram\per\kilogram}$.
\newline

\textbf{Visualization:}
%
\begin{figure}[!htp]
{\noindent\includegraphics[height=10cm, keepaspectratio]{figs/ads/ads_R-134a_activated_carbon_fiber_A-20_DubininAstakhov_3.pdf}}
\end{figure}
%

To generate the figure, the following refrigerant functions were selected:
\begin{itemize}
\item Vapor pressure: VaporPressure\_EoS1 - ID 1
\item Saturated liquid density: SaturatedLiquidDensity\_EoS1 - ID 1
\end{itemize}

The uncertainity of the experimental data is:
\begin{itemize}
\item Data source $\,\to\,$ Data was taken from figure
\item Pressure, relative, in \% $\,\to\,$ 0 Pa < p < 5 MPa: 0.1% full scale
\item Temperature, absolute, in $\si{\kelvin}$ $\,\to\,$ 0.15
\end{itemize}

The mean absolute percentage error (MAPE) between the experimental and calculated data results in 6.37\%.
\FloatBarrier
\newpage
%%%%%%%%%%%%%%%%%%%%%%%%%%%%%%%%%%%%%%%%%%%%%%%%%%%%%%%%%%%%%%%%%%%%%%%%%%%%%%%
