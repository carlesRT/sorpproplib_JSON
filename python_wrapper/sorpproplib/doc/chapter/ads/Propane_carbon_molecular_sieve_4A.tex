%%%%%%%%%%%%%%%%%%%%%%%%%%%%%%%%%%%%%%%%%%%%%%%%%%%%%%%%%%%%%%%%%%%%%%%%%%%%%%%
% Chapter 'Adsorption - Propane - carbon molecular sieve 4A'
%%%%%%%%%%%%%%%%%%%%%%%%%%%%%%%%%%%%%%%%%%%%%%%%%%%%%%%%%%%%%%%%%%%%%%%%%%%%%%%
\subsection{Carbon molecular sieve 4A}
%
%%%%%%%%%%%%%%%%%%%%%%%%%%%%%%%%%%%%%%%%%%%%%%%%%%%%%%%%%%%%%%%%%%%%%%%%%%%%%%%
%%%%%%%%%%%%%%%%%%%%%%%%%%%%%%%%%%%%%%%%%%%%%%%%%%%%%%%%%%%%%%%%%%%%%%%%%%%%%%%
\subsubsection{DubininAstakhov - ID 1}
%
\begin{tabular}[l]{|lp{11.5cm}|}
\hline
\addlinespace

\textbf{Sorbent:} & carbon molecular sieve \\
\textbf{Subtype:} & 4A \\
\textbf{Refrigerant:} & Propane \\
\textbf{Equation:} & DubininAstakhov \\
\textbf{ID:} & 1 \\
\textbf{Reference:} & Grande, Carlos A.; Silva, Viviana M.T.M.; Gigola, Carlos; Rodrigues, Alirio E. (2003): Adsorption of propane and propylene onto carbon molecular sieve. In: Carbon 41 (13), S. 2533–2545. DOI: 10.1016/S0008-6223(03)00304-X. \\
\textbf{Comment:} & None \\

\addlinespace
\hline
\end{tabular}
\newline

\textbf{Properties of sorbent:}
\newline
%
\begin{longtable}[l]{lll}
\toprule
\addlinespace
\textbf{Property} & \textbf{Unit} & \textbf{Value} \\
\addlinespace
\midrule
\endhead
\bottomrule
\endfoot
\bottomrule
\endlastfoot
\addlinespace

Diameter of pore & \si{\milli\meter} & 0.0003\\
Diameter of pellet & \si{\milli\meter} & 1.1\\
Length of extrudate & \si{\milli\meter} & 2.3-5\\
Porosity of pellet & - & 0.315\\
Pellet density & \si{\kilogram\per\cubic\meter} & 900\\

\addlinespace\end{longtable}

\textbf{Equation and parameters:}
\newline
%
Loading $w$ in $\si{\kilogram\per\kilogram}$ is calculated depending on pressure $p$ in $\si{\pascal}$, temperature $T$ in $\si{\kelvin}$, and vapor pressure $p_\mathrm{sat}$ in $\si{\pascal}$ by:
%
\begin{equation*}
\begin{split}
w &=& \begin{cases} W \rho_\mathrm{sat}^{\mathrm{liq}} & \quad \text{if flag} \geq 0 \\ W & \quad \text{else} \end{cases} & \quad\text{, and} \\
W &=& W_0 \exp \left( - \left( \nicefrac{A}{E} \right) ^{n} \right) & \quad\text{, and} \\
A &=& R T \ln \left( \nicefrac{p_\mathrm{sat}}{p} \right) & \quad\text{.} \\
\end{split}
\end{equation*}
%
The parameters of the equation are:
%
\begin{longtable}[l]{lll|lll}
\toprule
\addlinespace
\textbf{Par.} & \textbf{Unit} & \textbf{Value} &	\textbf{Par.} & \textbf{Unit} & \textbf{Value} \\
\addlinespace
\midrule
\endhead

\bottomrule
\endfoot
\bottomrule
\endlastfoot
\addlinespace

flag & - & -1.000000000e+00 & $n$ & - & 2.000000000e+00 \\
$E$ & $\si{\joule\per\mole}$ & 1.823000000e+04 & $W_0$ & $\si{\kilogram\per\kilogram}$ & 6.028470000e-02 \\

\addlinespace\end{longtable}

\textbf{Validity:}
\newline
Equation is approximately valid for $1834.0 \si{\pascal} \leq p \leq 290732.0 \si{\pascal}$,  $343.0 \si{\kelvin} \leq T \leq 423.0 \si{\kelvin}$, and $0.011958156 \si{\kilogram\per\kilogram} \leq w \leq 0.053757018 \si{\kilogram\per\kilogram}$.
\newline

\textbf{Visualization:}
%
\begin{figure}[!htp]
{\noindent\includegraphics[height=10cm, keepaspectratio]{figs/ads/ads_Propane_carbon_molecular_sieve_4A_DubininAstakhov_1.pdf}}
\end{figure}
%

To generate the figure, the following refrigerant functions were selected:
\begin{itemize}
\item Vapor pressure: VaporPressure\_EoS1 - ID 1
\item Saturated liquid density: SaturatedLiquidDensity\_EoS1 - ID 1
\end{itemize}

The uncertainity of the experimental data is:
\begin{itemize}
\item Data source $\,\to\,$ Data was taken from figure
\item Pressure, absolute, in $\si{\pascal}$ $\,\to\,$ 500
\item Temperature, absolute, in $\si{\kelvin}$ $\,\to\,$ 0.5
\end{itemize}

The mean absolute percentage error (MAPE) between the experimental and calculated data results in 2.9\%.
\FloatBarrier
\newpage
%%%%%%%%%%%%%%%%%%%%%%%%%%%%%%%%%%%%%%%%%%%%%%%%%%%%%%%%%%%%%%%%%%%%%%%%%%%%%%%
%%%%%%%%%%%%%%%%%%%%%%%%%%%%%%%%%%%%%%%%%%%%%%%%%%%%%%%%%%%%%%%%%%%%%%%%%%%%%%%
\subsubsection{Toth - ID 1}
%
\begin{tabular}[l]{|lp{11.5cm}|}
\hline
\addlinespace

\textbf{Sorbent:} & carbon molecular sieve \\
\textbf{Subtype:} & 4A \\
\textbf{Refrigerant:} & Propane \\
\textbf{Equation:} & Toth \\
\textbf{ID:} & 1 \\
\textbf{Reference:} & Grande, Carlos A.; Silva, Viviana M.T.M.; Gigola, Carlos; Rodrigues, Alirio E. (2003): Adsorption of propane and propylene onto carbon molecular sieve. In: Carbon 41 (13), S. 2533–2545. DOI: 10.1016/S0008-6223(03)00304-X. \\
\textbf{Comment:} & None \\

\addlinespace
\hline
\end{tabular}
\newline

\textbf{Properties of sorbent:}
\newline
%
\begin{longtable}[l]{lll}
\toprule
\addlinespace
\textbf{Property} & \textbf{Unit} & \textbf{Value} \\
\addlinespace
\midrule
\endhead
\bottomrule
\endfoot
\bottomrule
\endlastfoot
\addlinespace

Diameter of pore & \si{\milli\meter} & 0.0003\\
Diameter of pellet & \si{\milli\meter} & 1.1\\
Length of extrudate & \si{\milli\meter} & 2.3-5\\
Pore volume & \si{\milli\cubic\meter\per\gram} & 0.31\\
Porosity of pellet & - & 0.315\\
Pellet density & \si{\kilogram\per\cubic\meter} & 900\\

\addlinespace\end{longtable}

\textbf{Equation and parameters:}
\newline
%
Loading $w$ in $\si{\kilogram\per\kilogram}$ is calculated depending on pressure $p$ in $\si{\pascal}$ and temperature $T$ in $\si{\kelvin}$ by:
%
\begin{equation*}
\begin{split}
w &=& \frac{w_\mathrm{sat} b^{m} p}{\left( 1 + b^{r} p^{n} \right)^{\nicefrac{1}{n}}} & \quad\text{, and} \\
b &=& b_0 \exp\left( \frac{Q^{*}}{T} \right) & \quad\text{, and} \\
n &=& n_0 + \nicefrac{c}{T} & \quad\text{, and} \\
r &=& \begin{cases} n & \quad \text{if } r^{*} < 0 \\ r^{*}  & \quad \text{else} \end{cases} & \quad\text{.}
\end{split}
\end{equation*}
%
The parameters of the equation are:
%
\begin{longtable}[l]{lll|lll}
\toprule
\addlinespace
\textbf{Par.} & \textbf{Unit} & \textbf{Value} &	\textbf{Par.} & \textbf{Unit} & \textbf{Value} \\
\addlinespace
\midrule
\endhead

\bottomrule
\endfoot
\bottomrule
\endlastfoot
\addlinespace

$b_0$ & $\si{\per\pascal}$ & 1.547670750e-03 & $Q^{*}$ & $\si{\kelvin}$ & 1.378321189e+03 \\
$c$ & $\si{\kelvin}$ & 0.000000000e+00 & $r^{*}$ & - & 1.000000000e+00 \\
$m$ & - & 2.808988764e+00 & $w_\mathrm{sat}$ & $\si{\kilogram\per\kilogram}$ & 7.747842900e-02 \\
$n_0$ & - & 3.560000000e-01 & & & \\

\addlinespace\end{longtable}

\textbf{Validity:}
\newline
Equation is approximately valid for $2205.0 \si{\pascal} \leq p \leq 291114.0 \si{\pascal}$,  $343.0 \si{\kelvin} \leq T \leq 423.0 \si{\kelvin}$, and $0.011563997 \si{\kilogram\per\kilogram} \leq w \leq 0.053701768 \si{\kilogram\per\kilogram}$.
\newline

\textbf{Visualization:}
%
\begin{figure}[!htp]
{\noindent\includegraphics[height=10cm, keepaspectratio]{figs/ads/ads_Propane_carbon_molecular_sieve_4A_Toth_1.pdf}}
\end{figure}
%

To generate the figure, the following refrigerant functions were selected:
\begin{itemize}
\item Vapor pressure: VaporPressure\_EoS1 - ID 1
\item Saturated liquid density: SaturatedLiquidDensity\_EoS1 - ID 1
\end{itemize}

The uncertainity of the experimental data is:
\begin{itemize}
\item Data source $\,\to\,$ Data was taken from figure
\item Pressure, absolute, in $\si{\pascal}$ $\,\to\,$ 500
\item Temperature, absolute, in $\si{\kelvin}$ $\,\to\,$ 0.5
\end{itemize}

The mean absolute percentage error (MAPE) between the experimental and calculated data results in 2.92\%.
\FloatBarrier
\newpage
%%%%%%%%%%%%%%%%%%%%%%%%%%%%%%%%%%%%%%%%%%%%%%%%%%%%%%%%%%%%%%%%%%%%%%%%%%%%%%%
