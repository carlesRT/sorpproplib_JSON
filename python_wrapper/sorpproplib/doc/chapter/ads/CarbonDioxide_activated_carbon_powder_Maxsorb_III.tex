%%%%%%%%%%%%%%%%%%%%%%%%%%%%%%%%%%%%%%%%%%%%%%%%%%%%%%%%%%%%%%%%%%%%%%%%%%%%%%%
% Chapter 'Adsorption - CarbonDioxide - activated carbon powder Maxsorb III'
%%%%%%%%%%%%%%%%%%%%%%%%%%%%%%%%%%%%%%%%%%%%%%%%%%%%%%%%%%%%%%%%%%%%%%%%%%%%%%%
\subsection{Activated carbon powder Maxsorb III}
%
%%%%%%%%%%%%%%%%%%%%%%%%%%%%%%%%%%%%%%%%%%%%%%%%%%%%%%%%%%%%%%%%%%%%%%%%%%%%%%%
%%%%%%%%%%%%%%%%%%%%%%%%%%%%%%%%%%%%%%%%%%%%%%%%%%%%%%%%%%%%%%%%%%%%%%%%%%%%%%%
\subsubsection{Langmuir - ID 1}
%
\begin{tabular}[l]{|lp{11.5cm}|}
\hline
\addlinespace

\textbf{Sorbent:} & activated carbon powder \\
\textbf{Subtype:} & Maxsorb III \\
\textbf{Refrigerant:} & CarbonDioxide \\
\textbf{Equation:} & Langmuir \\
\textbf{ID:} & 1 \\
\textbf{Reference:} & Saha, Bidyut Baran; Jribi, Skander; Koyama, Shigeru; El-Sharkawy, Ibrahim I. (2011): Carbon Dioxide Adsorption Isotherms on Activated Carbons. In: J. Chem. Eng. Data 56 (5), S. 1974–1981. DOI: 10.1021/je100973t. \\
\textbf{Comment:} & None \\

\addlinespace
\hline
\end{tabular}
\newline

\textbf{Properties of sorbent:}
\newline
%
\begin{longtable}[l]{lll}
\toprule
\addlinespace
\textbf{Property} & \textbf{Unit} & \textbf{Value} \\
\addlinespace
\midrule
\endhead
\bottomrule
\endfoot
\bottomrule
\endlastfoot
\addlinespace

Diameter of pore & \si{\milli\meter} & 0.000002\\
Surface area & \si{\square\meter\per\gram} & 3150\\
Pore volume & \si{\milli\cubic\meter\per\gram} & 1.7\\

\addlinespace\end{longtable}

\textbf{Equation and parameters:}
\newline
%
Loading $w$ in $\si{\kilogram\per\kilogram}$ is calculated depending on pressure $p$ in $\si{\pascal}$ and temperature $T$ in $\si{\kelvin}$ by:
%
\begin{equation*}
\begin{split}
w &=& \frac{w_\mathrm{sat} K p}{1 + K p} & \quad\text{, and} \\
K &=& K_0 \exp \left( \frac{\Delta H}{R T} \right) & \quad\text{.}
\end{split}
\end{equation*}
%
The parameters of the equation are:
%
\begin{longtable}[l]{lll|lll}
\toprule
\addlinespace
\textbf{Par.} & \textbf{Unit} & \textbf{Value} &	\textbf{Par.} & \textbf{Unit} & \textbf{Value} \\
\addlinespace
\midrule
\endhead

\bottomrule
\endfoot
\bottomrule
\endlastfoot
\addlinespace

$\Delta H$ & $\si{\joule\per\mole}$ & 2.056000000e+04 & $w_\mathrm{sat}$ & $\si{\kilogram\per\kilogram}$ & 2.210000000e+00 \\
$K_0$ & $\si{\per\pascal}$ & 1.060000000e-10 & & & \\

\addlinespace\end{longtable}

\textbf{Validity:}
\newline
Equation is approximately valid for $97000.0 \si{\pascal} \leq p \leq 10164000.0 \si{\pascal}$,  $263.15 \si{\kelvin} \leq T \leq 343.15 \si{\kelvin}$, and $0.121 \si{\kilogram\per\kilogram} \leq w \leq 1.719 \si{\kilogram\per\kilogram}$.
\newline

\textbf{Visualization:}
%
\begin{figure}[!htp]
{\noindent\includegraphics[height=10cm, keepaspectratio]{figs/ads/ads_CarbonDioxide_activated_carbon_powder_Maxsorb_III_Langmuir_1.pdf}}
\end{figure}
%

To generate the figure, the following refrigerant functions were selected:
\begin{itemize}
\item Vapor pressure: VaporPressure\_EoS1 - ID 1
\item Saturated liquid density: SaturatedLiquidDensity\_EoS1 - ID 1
\end{itemize}

The uncertainity of the experimental data is:
\begin{itemize}
\item Data source $\,\to\,$ Data was taken from figure
\item Pressure, relative, in \% $\,\to\,$ 0 Pa < p < 1000000 Pa: 0.2 % full scale
\item Temperature, absolute, in $\si{\kelvin}$ $\,\to\,$ 0.1
\end{itemize}

The mean absolute percentage error (MAPE) between the experimental and calculated data results in 4.97\%.
\FloatBarrier
\newpage
%%%%%%%%%%%%%%%%%%%%%%%%%%%%%%%%%%%%%%%%%%%%%%%%%%%%%%%%%%%%%%%%%%%%%%%%%%%%%%%
%%%%%%%%%%%%%%%%%%%%%%%%%%%%%%%%%%%%%%%%%%%%%%%%%%%%%%%%%%%%%%%%%%%%%%%%%%%%%%%
\subsubsection{Toth - ID 1}
%
\begin{tabular}[l]{|lp{11.5cm}|}
\hline
\addlinespace

\textbf{Sorbent:} & activated carbon powder \\
\textbf{Subtype:} & Maxsorb III \\
\textbf{Refrigerant:} & CarbonDioxide \\
\textbf{Equation:} & Toth \\
\textbf{ID:} & 1 \\
\textbf{Reference:} & Saha, Bidyut Baran; Jribi, Skander; Koyama, Shigeru; El-Sharkawy, Ibrahim I. (2011): Carbon Dioxide Adsorption Isotherms on Activated Carbons. In: J. Chem. Eng. Data 56 (5), S. 1974–1981. DOI: 10.1021/je100973t. \\
\textbf{Comment:} & None \\

\addlinespace
\hline
\end{tabular}
\newline

\textbf{Properties of sorbent:}
\newline
%
\begin{longtable}[l]{lll}
\toprule
\addlinespace
\textbf{Property} & \textbf{Unit} & \textbf{Value} \\
\addlinespace
\midrule
\endhead
\bottomrule
\endfoot
\bottomrule
\endlastfoot
\addlinespace

Diameter of pore & \si{\milli\meter} & 0.000002\\
Surface area & \si{\square\meter\per\gram} & 3150\\
Pore volume & \si{\milli\cubic\meter\per\gram} & 1.07\\

\addlinespace\end{longtable}

\textbf{Equation and parameters:}
\newline
%
Loading $w$ in $\si{\kilogram\per\kilogram}$ is calculated depending on pressure $p$ in $\si{\pascal}$ and temperature $T$ in $\si{\kelvin}$ by:
%
\begin{equation*}
\begin{split}
w &=& \frac{w_\mathrm{sat} b^{m} p}{\left( 1 + b^{r} p^{n} \right)^{\nicefrac{1}{n}}} & \quad\text{, and} \\
b &=& b_0 \exp\left( \frac{Q^{*}}{T} \right) & \quad\text{, and} \\
n &=& n_0 + \nicefrac{c}{T} & \quad\text{, and} \\
r &=& \begin{cases} n & \quad \text{if } r^{*} < 0 \\ r^{*}  & \quad \text{else} \end{cases} & \quad\text{.}
\end{split}
\end{equation*}
%
The parameters of the equation are:
%
\begin{longtable}[l]{lll|lll}
\toprule
\addlinespace
\textbf{Par.} & \textbf{Unit} & \textbf{Value} &	\textbf{Par.} & \textbf{Unit} & \textbf{Value} \\
\addlinespace
\midrule
\endhead

\bottomrule
\endfoot
\bottomrule
\endlastfoot
\addlinespace

$b_0$ & $\si{\per\pascal}$ & 1.170000000e-10 & $Q^{*}$ & $\si{\kelvin}$ & 2.449947872e+03 \\
$c$ & $\si{\kelvin}$ & 0.000000000e+00 & $r^{*}$ & - & -1.000000000e+00 \\
$m$ & - & 1.000000000e+00 & $w_\mathrm{sat}$ & $\si{\kilogram\per\kilogram}$ & 3.060000000e+00 \\
$n_0$ & - & 6.640000000e-01 & & & \\

\addlinespace\end{longtable}

\textbf{Validity:}
\newline
Equation is approximately valid for $97000.0 \si{\pascal} \leq p \leq 10164000.0 \si{\pascal}$,  $263.15 \si{\kelvin} \leq T \leq 343.15 \si{\kelvin}$, and $0.121 \si{\kilogram\per\kilogram} \leq w \leq 1.719 \si{\kilogram\per\kilogram}$.
\newline

\textbf{Visualization:}
%
\begin{figure}[!htp]
{\noindent\includegraphics[height=10cm, keepaspectratio]{figs/ads/ads_CarbonDioxide_activated_carbon_powder_Maxsorb_III_Toth_1.pdf}}
\end{figure}
%

To generate the figure, the following refrigerant functions were selected:
\begin{itemize}
\item Vapor pressure: VaporPressure\_EoS1 - ID 1
\item Saturated liquid density: SaturatedLiquidDensity\_EoS1 - ID 1
\end{itemize}

The uncertainity of the experimental data is:
\begin{itemize}
\item Data source $\,\to\,$ Data was taken from figure
\item Pressure, relative, in \% $\,\to\,$ 0 Pa < p < 1000000 Pa: 0.2 % full scale
\item Temperature, absolute, in $\si{\kelvin}$ $\,\to\,$ 0.1
\end{itemize}

The mean absolute percentage error (MAPE) between the experimental and calculated data results in 2.79\%.
\FloatBarrier
\newpage
%%%%%%%%%%%%%%%%%%%%%%%%%%%%%%%%%%%%%%%%%%%%%%%%%%%%%%%%%%%%%%%%%%%%%%%%%%%%%%%
