%%%%%%%%%%%%%%%%%%%%%%%%%%%%%%%%%%%%%%%%%%%%%%%%%%%%%%%%%%%%%%%%%%%%%%%%%%%%%%%
% Chapter 'Adsorption - Methane - activated carbon fiber A-20'
%%%%%%%%%%%%%%%%%%%%%%%%%%%%%%%%%%%%%%%%%%%%%%%%%%%%%%%%%%%%%%%%%%%%%%%%%%%%%%%
\subsection{Activated carbon fiber A-20}
%
%%%%%%%%%%%%%%%%%%%%%%%%%%%%%%%%%%%%%%%%%%%%%%%%%%%%%%%%%%%%%%%%%%%%%%%%%%%%%%%
%%%%%%%%%%%%%%%%%%%%%%%%%%%%%%%%%%%%%%%%%%%%%%%%%%%%%%%%%%%%%%%%%%%%%%%%%%%%%%%
\subsubsection{DubininAstakhov - ID 1}
%
\begin{tabular}[l]{|lp{11.5cm}|}
\hline
\addlinespace

\textbf{Sorbent:} & activated carbon fiber \\
\textbf{Subtype:} & A-20 \\
\textbf{Refrigerant:} & Methane \\
\textbf{Equation:} & DubininAstakhov \\
\textbf{ID:} & 1 \\
\textbf{Reference:} & Rahman, Kazi Afzalur; Chakraborty, Anutosh; Saha, Bidyut Baran; Ng, Kim Choon (2012): On thermodynamics of methane+carbonaceous materials adsorption. In: International Journal of Heat and Mass Transfer 55 (4), S. 565–573. DOI: 10.1016/j.ijheatmasstransfer.2011.10.056. \\
\textbf{Comment:} & See original literature: Use low-level interface for calculations as special form of density of adsorpt (i.e., 1/rho\_adsorpt = 2.3677e-3 * exp(0.0043* (T - 111.67)) in kg/m3) and vapor pressure (i.e., p\_sat = p\_crit *(T / T\_crit)\^2 in Pa if T > T\_crit) are required; inverse functions may not work anymore. \\

\addlinespace
\hline
\end{tabular}
\newline

\textbf{Properties of sorbent:}
\newline
%
\begin{longtable}[l]{lll}
\toprule
\addlinespace
\textbf{Property} & \textbf{Unit} & \textbf{Value} \\
\addlinespace
\midrule
\endhead
\bottomrule
\endfoot
\bottomrule
\endlastfoot
\addlinespace

Surface area & \si{\square\meter\per\gram} & 2206\\
Pore volume & \si{\milli\cubic\meter\per\gram} & 1.01\\

\addlinespace\end{longtable}

\textbf{Equation and parameters:}
\newline
%
Loading $w$ in $\si{\kilogram\per\kilogram}$ is calculated depending on pressure $p$ in $\si{\pascal}$, temperature $T$ in $\si{\kelvin}$, and vapor pressure $p_\mathrm{sat}$ in $\si{\pascal}$ by:
%
\begin{equation*}
\begin{split}
w &=& \begin{cases} W \rho_\mathrm{sat}^{\mathrm{liq}} & \quad \text{if flag} \geq 0 \\ W & \quad \text{else} \end{cases} & \quad\text{, and} \\
W &=& W_0 \exp \left( - \left( \nicefrac{A}{E} \right) ^{n} \right) & \quad\text{, and} \\
A &=& R T \ln \left( \nicefrac{p_\mathrm{sat}}{p} \right) & \quad\text{.} \\
\end{split}
\end{equation*}
%
The parameters of the equation are:
%
\begin{longtable}[l]{lll|lll}
\toprule
\addlinespace
\textbf{Par.} & \textbf{Unit} & \textbf{Value} &	\textbf{Par.} & \textbf{Unit} & \textbf{Value} \\
\addlinespace
\midrule
\endhead

\bottomrule
\endfoot
\bottomrule
\endlastfoot
\addlinespace

flag & - & 1.000000000e+00 & $n$ & - & 1.510000000e+00 \\
$E$ & $\si{\joule\per\mole}$ & 6.198400000e+03 & $W_0$ & $\si{\cubic\meter\per\kilogram}$ & 7.170000000e-04 \\

\addlinespace\end{longtable}

\textbf{Validity:}
\newline
Equation is approximately valid for $278.0 \si{\kelvin} \leq T \leq 348.0 \si{\kelvin}$.
\newline

\textbf{Visualization:}
%
\newline
No experimental data exists. Thus, isotherm is not visualized!
%

\FloatBarrier
\newpage
%%%%%%%%%%%%%%%%%%%%%%%%%%%%%%%%%%%%%%%%%%%%%%%%%%%%%%%%%%%%%%%%%%%%%%%%%%%%%%%
