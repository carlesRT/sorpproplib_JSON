%%%%%%%%%%%%%%%%%%%%%%%%%%%%%%%%%%%%%%%%%%%%%%%%%%%%%%%%%%%%%%%%%%%%%%%%%%%%%%%
% Chapter 'Absorption - Cyclohexene - inoic liquid [EMIM]+[(CF3SO2)2N]-'
%%%%%%%%%%%%%%%%%%%%%%%%%%%%%%%%%%%%%%%%%%%%%%%%%%%%%%%%%%%%%%%%%%%%%%%%%%%%%%%
\subsection{Inoic liquid [EMIM]+[(CF3SO2)2N]-}
%
%%%%%%%%%%%%%%%%%%%%%%%%%%%%%%%%%%%%%%%%%%%%%%%%%%%%%%%%%%%%%%%%%%%%%%%%%%%%%%%
%%%%%%%%%%%%%%%%%%%%%%%%%%%%%%%%%%%%%%%%%%%%%%%%%%%%%%%%%%%%%%%%%%%%%%%%%%%%%%%
\subsubsection{NrtlTemperatureDg - ID 1}
%
\begin{tabular}[l]{|lp{11.5cm}|}
\hline
\addlinespace

\textbf{Sorbent:} & inoic liquid \\
\textbf{Subtype:} & [EMIM]+[(CF3SO2)2N]- \\
\textbf{Refrigerant:} & Cyclohexene \\
\textbf{Equation:} & NrtlTemperatureDg \\
\textbf{ID:} & 1 \\
\textbf{Reference:} & Kato, Ryo; Krummen, Michael; Gmehling, Jürgen (2004): Measurement and correlation of vapor–liquid equilibria and excess enthalpies of binary systems containing ionic liquids and hydrocarbons. In: Fluid Phase Equilibria 224 (1), S. 47–54. DOI: 10.1016/j.fluid.2004.05.009. \\
\textbf{Comment:} & None \\

\addlinespace
\hline
\end{tabular}
\newline

\textbf{Equation and parameters:}
\newline
%
Pressure $p$ in $\si{\pascal}$ is calculated depending on molar fraction of refrigerant in the liquid phase $x_1$ in $\si{\mole\per\mole}$, temperature $T$ in $\si{\kelvin}$, and vapor pressure $p_\mathrm{sat,1}$ in $\si{\pascal}$ by:
%
\begin{equation*}
\begin{split}
p &=& \gamma_1 x_1 p_\mathrm{sat,1} & \quad\text{, and} \\
\gamma_1 &=& \exp \left( x_2^2 \left( \tau_{21} \left( \frac{G_{21}}{x_1 + x_2 G_{21}} \right) ^2 + \tau_{12} \frac{G_{12}}{\left( x_2 + x_1 G_{12} \right) ^2}\right) \right) & \quad\text{, and} \\
G_{12} &=& \exp \left( -\alpha_{12} \tau_{12} \right) & \quad\text{, and} \\
G_{21} &=& \exp \left( -\alpha_{21} \tau_{21} \right) & \quad\text{, and} \\
\tau_{12} &=& \nicefrac{\Delta g_{12}}{R T} & \quad\text{, and} \\
\tau_{21} &=& \nicefrac{\Delta g_{21}}{R T} & \quad\text{, and} \\
\Delta g_{12} &=& a_{12} + b_{12} T & \quad\text{, and} \\
\Delta g_{21} &=& a_{21} + b_{21} T & \quad\text{, and} \\
x_2 &=& 1 - x_1  & \quad\text{.} \\
\end{split}
\end{equation*}
%
The parameters of the equation are:
%
\begin{longtable}[l]{lll|lll}
\toprule
\addlinespace
\textbf{Par.} & \textbf{Unit} & \textbf{Value} &	\textbf{Par.} & \textbf{Unit} & \textbf{Value} \\
\addlinespace
\midrule
\endhead

\bottomrule
\endfoot
\bottomrule
\endlastfoot
\addlinespace

$\alpha_{12}$ & - & 2.000000000e-01 & $\alpha_{21}$ & - & 2.000000000e-01 \\
$a_{12}$ & $\si{\joule\per\mole}$ & 4.964320000e+04 & $a_{21}$ & $\si{\joule\per\mole}$ & 1.132440000e+04 \\
$b_{12}$ & $\si{\joule\per\mole\per\kelvin}$ & 6.886000000e+01 & $b_{21}$ & $\si{\joule\per\mole\per\kelvin}$ & -2.960000000e+01 \\

\addlinespace\end{longtable}

\textbf{Validity:}
\newline
Equation is approximately valid for $353.15 \si{\kelvin} \leq T \leq 353.15 \si{\kelvin}$.
\newline

\textbf{Visualization:}
%
\newline
No experimental data exists. Thus, isotherm is not visualized!
%

\FloatBarrier
\newpage
%%%%%%%%%%%%%%%%%%%%%%%%%%%%%%%%%%%%%%%%%%%%%%%%%%%%%%%%%%%%%%%%%%%%%%%%%%%%%%%
