%%%%%%%%%%%%%%%%%%%%%%%%%%%%%%%%%%%%%%%%%%%%%%%%%%%%%%%%%%%%%%%%%%%%%%%%%%%%%%%
% Chapter 'Absorption - Water - NaOH-KOH-CsOH '
%%%%%%%%%%%%%%%%%%%%%%%%%%%%%%%%%%%%%%%%%%%%%%%%%%%%%%%%%%%%%%%%%%%%%%%%%%%%%%%
\subsection{Naoh-koh-csoh }
%
%%%%%%%%%%%%%%%%%%%%%%%%%%%%%%%%%%%%%%%%%%%%%%%%%%%%%%%%%%%%%%%%%%%%%%%%%%%%%%%
%%%%%%%%%%%%%%%%%%%%%%%%%%%%%%%%%%%%%%%%%%%%%%%%%%%%%%%%%%%%%%%%%%%%%%%%%%%%%%%
\subsubsection{Duehring - ID 1}
%
\begin{tabular}[l]{|lp{11.5cm}|}
\hline
\addlinespace

\textbf{Sorbent:} & NaOH-KOH-CsOH \\
\textbf{Subtype:} &  \\
\textbf{Refrigerant:} & Water \\
\textbf{Equation:} & Duehring \\
\textbf{ID:} & 1 \\
\textbf{Reference:} & Herold, Keith E.; Radermacher, Reinhard; Howe, Lawrence; Erickson, Donald C. (1991): Development of an absorption heat pump water heater using an aqueous ternary hydroxide working fluid. In: International Journal of Refrigeration 14 (3), S. 156–167. DOI: 10.1016/0140-7007(91)90070-W. \\
\textbf{Comment:} & None \\

\addlinespace
\hline
\end{tabular}
\newline

\textbf{Equation and parameters:}
\newline
%
Pressure $p$ in $\si{\pascal}$ is calculated depending on concentration $X$ in $\si{\kilogram\per\kilogram}$ and  temperature $T$ in $\si{\kelvin}$ by:
%
\begin{equation*}
\begin{split}
p &=& \nicefrac{1}{r} 10 ^ { C + \frac{D}{T_\mathrm{ref}} + \frac{E}{T_\mathrm{ref}^{2}} } & \quad\text{, and} \\
T_\mathrm{ref} &=& q + \frac{\left( n T + m - B \right)}{A}  & \quad\text{, and} \\
A &=& \sum_{i=0}^{3} a_0 X_\mathrm{cor} ^{i}  & \quad\text{, and} \\
B &=& \sum_{i=0}^{3} b_0 X_\mathrm{cor} ^{i}  & \quad\text{, and} \\
X_\mathrm{cor} &=& 100 X  & \quad\text{.} \\
\end{split}
\end{equation*}
%
The parameters of the equation are:
%
\begin{longtable}[l]{lll|lll}
\toprule
\addlinespace
\textbf{Par.} & \textbf{Unit} & \textbf{Value} &	\textbf{Par.} & \textbf{Unit} & \textbf{Value} \\
\addlinespace
\midrule
\endhead

\bottomrule
\endfoot
\bottomrule
\endlastfoot
\addlinespace

$a_0$ & - & 6.164233723e+00 & $b_0$ & - & -5.380343163e+01 \\
$a_1$ & - & -2.746665026e-01 & $b_1$ & - & 5.004848451e+00 \\
$a_2$ & - & 4.916023734e-03 & $b_2$ & - & -1.228273028e-01 \\
$a_3$ & - & -2.859098259e-05 & $b_3$ & - & 1.096142341e-03 \\
$C$ & - & 6.427154896e+00 & $D$ & $\si{\kelvin}$ & -1.208919437e+03 \\
$m$ & - & 0.000000000e+00 & $E$ & $\si{\square\kelvin}$ & -1.661599630e+05 \\
$n$ & - & 1.000000000e+00 & $q$ & - & 2.731500000e+02 \\
$r$ & $\si{\per\pascal}$ & 1.000000000e-03 & & & \\

\addlinespace\end{longtable}

\textbf{Validity:}
\newline
Equation is approximately valid for $282.15 \si{\kelvin} \leq T \leq 443.15 \si{\kelvin}$.
\newline

\textbf{Visualization:}
%
\newline
No experimental data exists. Thus, isotherm is not visualized!
%

\FloatBarrier
\newpage
%%%%%%%%%%%%%%%%%%%%%%%%%%%%%%%%%%%%%%%%%%%%%%%%%%%%%%%%%%%%%%%%%%%%%%%%%%%%%%%
