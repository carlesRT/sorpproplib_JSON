%%%%%%%%%%%%%%%%%%%%%%%%%%%%%%%%%%%%%%%%%%%%%%%%%%%%%%%%%%%%%%%%%%%%%%%%%%%%%%%
% Chapter 'Usage of SorpPropLib'
%%%%%%%%%%%%%%%%%%%%%%%%%%%%%%%%%%%%%%%%%%%%%%%%%%%%%%%%%%%%%%%%%%%%%%%%%%%%%%%
\chapter{Usage of SorpPropLib}
\label{cha:usage}
%
SorpPropLib can currently be used from 6 programming environments: \textit{C/C++}, \textit{Python}, \textit{Matlab}, \textit{Modelica}, \textit{LabVIEW}, and \textit{Excel}. The usage of SorpPropLib in the programming environments is briefly explained below.
%%%%%%%%%%%%%%%%%%%%%%%%%%%%%%%%%%%%%%%%%%%%%%%%%%%%%%%%%%%%%%%%%%%%%%%%%%%%%%%
%%%%%%%%%%%%%%%%%%%%%%%%%%%%%%%%%%%%%%%%%%%%%%%%%%%%%%%%%%%%%%%%%%%%%%%%%%%%%%%
% Chapter 'C/C++ and compiling the DLL'
%%%%%%%%%%%%%%%%%%%%%%%%%%%%%%%%%%%%%%%%%%%%%%%%%%%%%%%%%%%%%%%%%%%%%%%%%%%%%%%
\section{\textit{C/C++} and compiling DLL}
\label{cha:usage:c}
%
The \textit{C} code is in the folder "SorpPropLib\textbackslash c\_code\textbackslash". Examples of using the \textit{C} code are in the folder "SorpPropLib\textbackslash c\_code\textbackslash src\textbackslash", with file names beginning with "test\_".

Since the equilibrium models are implemented in \textit{C}, either the \textit{C} code (see examples) or a pre-compiled DLL (see example "test\_workingPair\_DLL.c") can be used directly. The pre-compiled DLLs are in the folder "SorpPropLib\textbackslash c\_code\textbackslash lib\textbackslash". Currently, DLLs are pre-compiled for the following systems/architectures: Windows 32bit, Windows 64bit, and Linux Centos 64bit. If the pre-compiled DLLs are not sufficient, makefiles are available for (1) the \textit{C} compiler from Microsoft Visual Studios and (2) the \textit{C} compiler from MinGW or GNU to compile user-specific DLLs. The makefiles are to be used as follows:
%
\begin{enumerate}
	\item \underline{Makefile for Microsoft Visual Studios:}
	\begin{enumerate}
		\item Replace the contents of the file "makefile" in the folder "SorpPropLib\textbackslash c\_code\textbackslash" with the contents of the file "makefile\_MSVS"
		\item Make sure that the variable "BUILD\_RELEASE" in the "makefile" file is set to "YES"
		\item Open the developer prompt for Microsoft Visual Studios
		\item Set the correct architecture (32bit or 64bit) by running the appropriate bat file (see the comment at the beginning of the makefile "makefile\_MSVS"; you may have to adapt the path to your installation folder of Microsoft Visual Studios)
		\item Navigate to the folder "SorpPropLib\textbackslash c\_code\textbackslash" (change drive by typing, e.g., "C:" or "D:"; change directory by typing "cd D:\textbackslash YourPath")
		\item Compile the DLL by executing the command "NMAKE all /C" (if all files are to be rebuilt, use the command "NMAKE all /A /C")
		\item The DLL is saved in the folder "SorpPropLib\textbackslash c\_code\textbackslash lib\textbackslash"
	\end{enumerate}
	\item \underline{Makefile for MinGW / GNU:}
	\begin{enumerate}
		\item Replace the content of the file "makefile" in the folder "SorpPropLib\textbackslash c\_code\textbackslash" with the content of the file "makefile\_MINGW"
		\item Make sure that the variable "BUILD\_RELEASE" is set to "YES" and the variable "TYPE\_SYSTEM" is set according to the system
		\item Open the command prompt under Windows or the console under Linux
		\item Navigate to the folder "SorpPropLib\textbackslash c\_code\textbackslash" (change drive by typing, e.g., "C:" or "D:"; change directory by typing "cd D:\textbackslash YourPath")
		\item Copy the path to the file "mingw32-make.exe", which corresponds to the required architecture (32bit or 64bit):
		\begin{enumerate}
			\item Example for 32bit: "C:\textbackslash MinGW32bit\textbackslash bin\textbackslash mingw32-make.exe"
			\item Example for 64bit: "C:\textbackslash MinGW64bit\textbackslash bin\textbackslash mingw32-make.exe"
		\end{enumerate}
		\item Compile the DLL by executing the command "’path to correct make.exe’ all” (if all files are to be rebuilt, use the command ”’path to correct make.exe’ all -B”)
		\item The DLL is saved in the folder “SorpPropLib\textbackslash c\_code\textbackslash lib\textbackslash”
	\end{enumerate}
\end{enumerate}
%
%%%%%%%%%%%%%%%%%%%%%%%%%%%%%%%%%%%%%%%%%%%%%%%%%%%%%%%%%%%%%%%%%%%%%%%%%%%%%%%
%%%%%%%%%%%%%%%%%%%%%%%%%%%%%%%%%%%%%%%%%%%%%%%%%%%%%%%%%%%%%%%%%%%%%%%%%%%%%%%
% Chapter 'Python'
%%%%%%%%%%%%%%%%%%%%%%%%%%%%%%%%%%%%%%%%%%%%%%%%%%%%%%%%%%%%%%%%%%%%%%%%%%%%%%%
\section{\textit{Python}}
\label{cha:usage:python}
%
The \textit{Python} interface is in the folder “SorpPropLib\textbackslash python\_wrapper\textbackslash ”. The required (pre)compiled DLL is to be stored in “SorpPropLib\textbackslash python\_wrapper\textbackslash sorp-proplib\textbackslash data\textbackslash 'CorrectSystemArchitecture'”. Examples for using the \textit{Python} interface are in the folder “SorpProp-Lib\textbackslash python\_wrapper\textbackslash ”, with file names starting with “example\_”:
\begin{itemize}
	\item \underline{example\_generate\_json\_file:} Generates the JSON database from CSV files in UTF8 format (see chapter \ref{cha:extensions})
	\item \underline{example\_generate\_content\_list:} Generates Excel files in the folder “SorpPropLib” containing all implemented equilibrium approaches for absorption and adsorption work pairs as well as all implemented refrigerant functions
	\item \underline{example\_generate\_manual:} Automatically generates LaTeX files in the folder "SorpPropLib\textbackslash python\_wrapper\textbackslash sorpproplib\textbackslash doc\textbackslash " to compile this manual
	\item \underline{example\_calling\_DLL:} Demonstrates all wrapper functions available in \textit{Python}
	\item \underline{example\_generate\_plots:} Demonstrates all implemented functions for visualizing equilibrium data
\end{itemize}
\textbf{\underline{Important notes:}} The \textit{Python} interface was developed using the \textit{Anaconda Python distibutio}n and using the additional package "CoolProp" to calculate refrigerant properties. Besides, the example files can be executed directly with \textit{Python} because the paths to the DLL and the JSON database are stored as relative paths in the example files.
%%%%%%%%%%%%%%%%%%%%%%%%%%%%%%%%%%%%%%%%%%%%%%%%%%%%%%%%%%%%%%%%%%%%%%%%%%%%%%%
%%%%%%%%%%%%%%%%%%%%%%%%%%%%%%%%%%%%%%%%%%%%%%%%%%%%%%%%%%%%%%%%%%%%%%%%%%%%%%%
% Chapter 'Matlab'
%%%%%%%%%%%%%%%%%%%%%%%%%%%%%%%%%%%%%%%%%%%%%%%%%%%%%%%%%%%%%%%%%%%%%%%%%%%%%%%
\section{\textit{Matlab}}
\label{cha:usage:matlab}
%
The \textit{Matlab} interface is in the folder “SorpPropLib\textbackslash matlab\_wrapper\textbackslash ”. The required (pre)compiled DLL and the JSON database are to be stored in “SorpPropLib\textbackslash matlab\_wrapper\textbackslash  data\textbackslash 'CorrectSystemArchitecture'”. Examples for using the \textit{Matlab} interface are in the folder “SorpPropLib\textbackslash matlab\_wrapper\textbackslash ”, with file names starting with “minimal\_example\_”:
\begin{itemize}
	\item \underline{minimal\_example\_absorption\_activity:} Demonstrates all available equilibrium functions of absorption work pairs based on activity coefficients (e.g., Wilson)
	\item \underline{minimal\_example\_absorption\_conventional:} Demonstrates all available equilibrium functions of absorption work pairs based on conventional approaches (e.g., Antoine or Dühring)
	\item \underline{minimum\_example\_absorption\_mixing:} Demonstrates all available equilibrium functions of absorption work pairs based on mixing rules of cubic equations of state
	\item \underline{minimum\_example\_adsorption\_surface:} Demonstrates all available equilibrium functions of adsorption work pairs based on the surface approach (e.g., Toth)
	\item \underline{minimal\_example\_adsorption\_surface\_vapor:} Demonstrates all available equilibrium functions of adsorption working pairs based on the surface approach and using the vapor pressure as an additional function argument (e.g., Freundlich)
	\item \underline{minimal\_example\_adsorption\_volumetric:} Demonstrates all available equilibrium functions of adsorption working pairs based on the volumetric approach and using the vapor pressure and density of the adsorpt as additional function arguments (e.g., Dubinin-Astakhov)
\end{itemize}
\textbf{\underline{Important note:}} The example files can be executed directly with \textit{Matlab} since the paths to the DLL and the JSON database are stored as relative paths in the example files.
%%%%%%%%%%%%%%%%%%%%%%%%%%%%%%%%%%%%%%%%%%%%%%%%%%%%%%%%%%%%%%%%%%%%%%%%%%%%%%%
%%%%%%%%%%%%%%%%%%%%%%%%%%%%%%%%%%%%%%%%%%%%%%%%%%%%%%%%%%%%%%%%%%%%%%%%%%%%%%%
% Chapter 'Modelica'
%%%%%%%%%%%%%%%%%%%%%%%%%%%%%%%%%%%%%%%%%%%%%%%%%%%%%%%%%%%%%%%%%%%%%%%%%%%%%%%
\section{\textit{Modelica}}
\label{cha:usage:modelica}
%
The \textit{Modelica} interface is in the folder "SorpProp-Lib\textbackslash modelica\_wrapper\textbackslash SorpPropLib\textbackslash ". The required (pre)compiled DLL and the JSON database are to be stored under "SorpPropLib\textbackslash modelica\_wrapper\textbackslash SorpPropLib\textbackslash Resources\textbackslash Library\textbackslash 'CorrectSystemArchitecture'" and "SorpPropLib\textbackslash modelica\_wrapper\textbackslash SorpPropLib\textbackslash Resources\textbackslash Data" respectively. Examples for using the \textit{Modelica} interface can be found in the \textit{Modelica} library in the packages "SorpPropLib.DirectFunctionCals.Tester" or "SorpPropLib.WorkingPair.Tester":
\begin{itemize}
	\item \underline{Test\_WPair\_refrigerant:} Demonstrates all available refrigerant functions
	\item \underline{Test\_WPair\_absorption\_activity:} Demonstrates all available equilibrium functions of absorption working pairs based on activity coefficients (e.g., Wilson)
	\item \underline{Test\_WPair\_absorption\_conventional:} Demonstrates all available equilibrium functions of absorption work pairs based on conventional approaches (e.g., Antoine or Dühring)
	\item \underline{Test\_WPair\_absorption\_mixing:} Demonstrates all available equilibrium functions of absorption working pairs based on mixing rules of cubic equations of state
	\item \underline{Test\_WPair\_adsorption\_surface:} Demonstrates all available equilibrium functions of adsorption work pairs starting based on the surface approach (e.g., Toth)
	\item \underline{Test\_WPair\_adsorption\_surface\_vapor:} Demonstrates all available equilibrium functions of adsorption working pairs based on the surface approach and using the vapor pressure as an additional function argument (e.g., Freundlich)
	\item \underline{Test\_WPair\_adsorption\_volumetric:} Demonstrates all available equilibrium functions of adsorption working pairs based on the volumetric approach and using the vapor pressure and density of the adsorpt as additional function arguments (e.g., Dubinin-Astakhov)
\end{itemize}
\textbf{\underline{Important note:}} The example files \textbf{\underline{cannot be executed directly}} in \textit{Modelica} because the path to the JSON database is not specified as a relative path. Therefore, the variable “path\_db” must be adjusted to execute the sample files.
%%%%%%%%%%%%%%%%%%%%%%%%%%%%%%%%%%%%%%%%%%%%%%%%%%%%%%%%%%%%%%%%%%%%%%%%%%%%%%%
%%%%%%%%%%%%%%%%%%%%%%%%%%%%%%%%%%%%%%%%%%%%%%%%%%%%%%%%%%%%%%%%%%%%%%%%%%%%%%%
% Chapter 'LabVIEW'
%%%%%%%%%%%%%%%%%%%%%%%%%%%%%%%%%%%%%%%%%%%%%%%%%%%%%%%%%%%%%%%%%%%%%%%%%%%%%%%
\section{\textit{LabVIEW}}
\label{cha:usage:labview}
%
The \textit{LabVIEW} interface is in the folder “SorpPropLib\textbackslash labview\_wrapper\textbackslash ”. The required (pre)compiled DLL and the JSON database are to be saved under “SorpPropLib\textbackslash labview\_wrapper\textbackslash data\textbackslash”. Examples for using the \textit{LabVIEW} interface can be found in the \textit{LabVIEW} project under "SorpPropLib.Examples" or under "SorpPropLib.Sub-libraries.direct\_call.Examples" and "SorpPropLib.Sub-libraries. struct\_call.Examples":
\begin{itemize}
	\item \underline{plot\_isosteric\_chart:} Demonstrates how to plot equilibrium data in an isosteric chart.
	\item \underline{Test\_abs\_activity:} Demonstrates all available equilibrium functions of absorption work pairs based on activity coefficients (e.g., Wilson)
	\item \underline{Test\_abs\_conventional:} Demonstrates all available equilibrium functions of absorption work pairs based on conventional approaches (e.g., Antoine or Dühring)
	\item \underline{Test\_abs\_mixing:} Demonstrates all available equilibrium functions of absorption working pairs based on mixing rules of cubic equations of state
	\item \underline{Test\_ads\_surface:} Demonstrates all available equilibrium functions of adsorption work pairs starting based on the surface approach (e.g., Toth)
	\item \underline{Test\_ads\_surface\_vapor:} Demonstrates all available equilibrium functions of adsorption working pairs based on the surface approach and using the vapor pressure as an additional function argument (e.g., Freundlich)
	\item \underline{Test\_ads\_volumetric:} Demonstrates all available equilibrium functions of adsorption working pairs based on the volumetric approach and using the vapor pressure and density of the adsorpt as additional function arguments (e.g., Dubinin-Astakhov)
\end{itemize}
\textbf{\underline{Important note:}} The example files \textbf{\underline{cannot be executed directly}} in \textit{LabVIEW} because the path to the JSON database is not specified as a relative path. Therefore, the variable “path\_db” must be adjusted to execute the example files.
%%%%%%%%%%%%%%%%%%%%%%%%%%%%%%%%%%%%%%%%%%%%%%%%%%%%%%%%%%%%%%%%%%%%%%%%%%%%%%%
%%%%%%%%%%%%%%%%%%%%%%%%%%%%%%%%%%%%%%%%%%%%%%%%%%%%%%%%%%%%%%%%%%%%%%%%%%%%%%%
% Chapter 'Excel'
%%%%%%%%%%%%%%%%%%%%%%%%%%%%%%%%%%%%%%%%%%%%%%%%%%%%%%%%%%%%%%%%%%%%%%%%%%%%%%%
\section{\textit{Excel}}
\label{cha:usage:excel}
%
The \textit{Excel} interface is in the folder “SorpPropLib\textbackslash excel\_wrapper\textbackslash ”. The required (pre)compiled DLL and the JSON database are to be stored under “SorpPropLib\textbackslash excel\_wrapper\textbackslash data\textbackslash”. Examples for using the \textit{Excel} interface can be found in the folder “SorpPropLib\textbackslash excel\_wrapper\textbackslash ”, with file names starting with “minimal\_example\_”:
\begin{itemize}
	\item \underline{minimal\_example\_dll\_functions:} Demonstrates all wrapper functions that are available in \textit{Excel}
	\item \underline{minimal\_example\_isosteric\_chart\_adsorption:} Demonstrates how to visualize adsorption equilibrium data in an isosteric chart
\end{itemize}
\textbf{\underline{Important note:}} The example files \textbf{\underline{cannot be executed directly}} in \textit{Excel} because the path to the DLL cannot be specified as a relative path but must be hardcoded. Therefore, the path to the DLL must be corrected before executing the \textit{Excel} files:
\begin{enumerate}
	\item Open the file "DirectFunctionCalls.bas" in the folder "SorpPropLib\textbackslash excel\_wrap-per\textbackslash vba\textbackslash ", e.g., with Notepad
	\item Search and replace all “lib ‘Path to DLL’” with “lib ‘Corrected path to DLL’”
	\item Open an \textit{Excel} example and allow macros
	\item If developer tools are not yet activated in \textit{Excel}, activate them under File -> Options -> Customize ribbon -> Check developer tools
	\item Open the application “Visual Basic” under Developer tools
	\item Delete module “DirectFunctionCalls” and add the module “DirectFunctionCalls” adapted before, which contains the correct path to DLL
	\item Recalculate a formula in \textit{Excel} so that all functions are updated
\end{enumerate}
%%%%%%%%%%%%%%%%%%%%%%%%%%%%%%%%%%%%%%%%%%%%%%%%%%%%%%%%%%%%%%%%%%%%%%%%%%%%%%%
%%%%%%%%%%%%%%%%%%%%%%%%%%%%%%%%%%%%%%%%%%%%%%%%%%%%%%%%%%%%%%%%%%%%%%%%%%%%%%%
% Chapter 'Known problems'
%%%%%%%%%%%%%%%%%%%%%%%%%%%%%%%%%%%%%%%%%%%%%%%%%%%%%%%%%%%%%%%%%%%%%%%%%%%%%%%
\section{Known problems}
\label{cha:usage:problems}
%
The following problems are currently known:
\begin{enumerate}
	\item \textit{Excel} wrapper is currently just working with \textit{Excel 2016} and not with \textit{Excel 365}.
	\item When running wrapper functions that rely on the approach ‘direct function calls’ (e.g., all wrapper functions in \textit{Excel}), RAM is only freed when the programming environment is closed and not when the calculation is finished. However, the error does not occur in the testers of the \textit{C} code. In addition, the error does not occur when wrapper functions based on the approach ‘working pair struct’ (i.e., mainly used in all examples) are executed.
	\item Some implemented fits represent the experimental data very poorly:
	\begin{enumerate}
		\item This happens mainly for absorption work pairs and in particular for absorption work pairs based on activity coefficients.
		\item The error is suspected to be in the parameterization of the working pairs or the extracted experimental data since all implemented equilibrium equations have been checked several times and for each equilibrium equations there are fits for work pairs that reproduce experimental data very well
		\item \textbf{\underline{Important note:}} Check the quality of the equilibrium approach before using it, e.g., with this manual (see chapters \ref{cha:refrigerants}-\ref{cha:absorption})
	\end{enumerate}
\end{enumerate}
%%%%%%%%%%%%%%%%%%%%%%%%%%%%%%%%%%%%%%%%%%%%%%%%%%%%%%%%%%%%%%%%%%%%%%%%%%%%%%%